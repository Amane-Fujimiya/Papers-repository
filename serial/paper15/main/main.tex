%%%%%%%%%%%%%%%%%%%%%%%%%%%%%%%%%%%%%%%%%%%%%%
% To select a journal, use its code for the 
% journal= option in the \documentclass command.
% The journal codes for this template are:

% Cambridge Prisms: pri
% Research Directions: rdi
% Computational Humanities Research: chr
%%%%%%%%%%%%%%%%%%%%%%%%%%%%%%%%%%%%%%%%%%%%%%
\documentclass[
  journal=large,
  manuscript=Review,
  year=2025,
  volume=39
]{cup-journal}
\usepackage[citecolor=blue,colorlinks=true]{hyperref}
\usepackage[utf8]{inputenc}
\usepackage{amsmath}
\usepackage[nopatch]{microtype}
\usepackage{booktabs}
\usepackage{amsthm}

\newtheorem{definition}{Definition}[section]
\newtheorem{theorem}{Theorem}[section]
\newtheorem{col}{Corollary}[subsection]
\newtheorem{conjecture}{Conjecture}[section]
\newtheorem{setting}{Setting}[section]
\newtheorem{proposition}[theorem]{Proposition}
%\newtheorem{lemma}[theorem]{Lemma}
\newtheorem{assumption}[theorem]{Assumption}
\newtheorem{assume}{Assumption}[subsection]
%\newtheorem{remark}[theorem]{Remark}
\newtheorem{hypothesis}{Hypothesis}[section]
%\newtheorem{axiom}{Axiom}[section]
\newtheorem{question}{Question}[section]
%\newtheorem{example}{Example}[section]
\newtheorem{note}{Note}[section]
\newtheorem{principle}{Principle}[section]

\title{The philosophy of theoretical modelling}

\author{Bui Gia Khanh}
\affiliation{Department of Physics, Hanoi University of Science, Vietnam National University, Hanoi, Vietnam}
\email[F. Author]{fujimiyaamane@outlook.com}

%\author{Duong Ngoc Khoa}
%\affiliation{Department of Physics, Hanoi University of Science, Vietnam National University, Hanoi, Vietnam}
% \alsoaffiliation{Joint first authors}

%\author{Yogesh Verma}
%\affiliation{TBH}

%\author{F.T. Author}
%\affiliation{Fourth Division, Organization, City, Pincode, State, Country}

\addbibresource{ramanresearch.bib}

\keywords{Philosophy $|$ Modelling theory $|$ Mathematics $|$ Philosophy of Mathematics} %% First letter not capped

\begin{document}

\begin{abstract}
  Nothing for now. 
\end{abstract}

%Let us consider the path $\mathbf{r}(s)$ of which light takes between point $A$ and $B$. By constraints, we assume that we observe light at $A$, and if we move to $B$, we also observe light at $B$. Thus, there exists certain \textit{action} of which such object of light moves in between these observations. Denote $A=\mathbf{r}(0)$ and $B=\mathbf{r}(d)$ for any given path of length $d$, light then follow the path defined by 
%\begin{equation}
%  \ell [\mathbf{r}]:= \int^{d}_{0} n(\mathbf{r})\: ds
%\end{equation}
%for $n$ the quantity of which denotes of its optical \textbf{refractive index}, or rather intrusion of matter and interferences in light path. This quantity, per classical theory, is proportional to the time light takes to traverse the path, i.e. $\Delta t = \ell /c_{0}$. Fermat's principle then simply state the following: 


\clearpage

\section{Conclusion}

\clearpage

\nocite{*}
\printbibliography

\clearpage 

\appendix

\section{More on particle-wave duality}



\end{document}
